\documentclass{article}
\usepackage[utf8]{inputenc}
\usepackage[T1]{fontenc}
\usepackage[ngerman]{babel}
\usepackage{mdframed}
\usepackage{amsmath,amssymb,mathrsfs,amsthm}

\theoremstyle{definition}
\newtheorem{Def}{Definition}
\newtheorem{Bsp}{Beispiel}

\theoremstyle{plain}
\newtheorem{Sat}{Satz}
\newtheorem{Lem}{Lemma}

\newmdenv[
  topline=false,
  bottomline=false,
  rightline=false,
  skipabove=\topsep,
  skipbelow=\topsep
]{siderule}


\newcommand{\ket}[1]{\left\vert #1\right>}
\newcommand{\bra}[1]{\left< #1\right\vert}
\def\Ell{\ensuremath{\mathscr{L}}}
\let\del\partial

\title{ETQFTs}

\author{Carsten Burgard}

\begin{document}

%% Outline
%% 
%% Einleitung (Carsten, Manuel)
Den meisten wissenschaftlich interessierten Menschen sind sowohl die
Quantenmechanik als auch die (allgemeine oder spezielle)
Relativitätstheorie ein Begriff. Diese beiden physikalischen Theorien
verkörpern in gewissem Sinne physikalische Grenzwerte, welche der
Alltagserfahrung von der uns umgebenden Physik oft eklatant
widersprechen. Der Grund hierfür ist, dass sich die Quantenmechanik
mit dem Grenzfall sehr kleiner Distanzen und Zeitspannen beschäftigt,
auf welchen die Gravitation oft keine Rolle spielt. Die (allgemeine)
Relativitätstheorie auf der anderen Seite modelliert makroskopische,
gravitative Vorgänge. Die Überlagerung dieser beiden Grenzfälle, also
gravitative Effekte auf mikrokopischer Ebene, ist nicht nur
experimentell sehr schwer zu beobachten, sondern erfordert auf der
theoretischen Seite auch eine Auseinandersetzung mit der Frage, wie
zwei konzeptionell so unterschiedliche Theorien wie Quantenmechanik
(QM) und Allgemeine Relativität (ART) auf eine konsistente Art und
Weise zusammengeführt werden können. Dieses Problem hat sich als
ausgesprochen schwierig herausgestellt, und so gibt es zum heutigen
Tage mehrere Ansätze, eine erfolgreiche theoretische Beschreibung von
Quantengravitation zu etablieren. Ein verhältnismäßig neues Konzept
für die Lösung dieses alten Problems sind Topologische
Quantenfeldtheorien (TQFTs), welche im Folgenden erläutert werden
sollen.

Um in diesem mathematischen Artikel einer diversen Leserschaft gerecht
zu werden, werden einige Passagen mit seitlichen Linien markiert --
diese dienen zur Veranschaulichung und können von lediglich
mathematisch interessierten Lesern übersprungen werden, während
physikalisch interessierte sie hoffentlich hilfreich finden werden
und lediglich an einer Anschauung interessierte Leser auch
ausschließlich die markierten Abschnitte lesen und die mathematischen
Passagen überspringen können,ohne dabei den Lesefluss zu behindern.

%% Raumzeit, Feld, Quantengravitation (Carsten, Jochen)
\begin{siderule}
In der QM werden physikalische Zustände oft durch Vektoren eines
Hilbertraums beschrieben. Im Unterschied zu einer klassischen
Betrachtung sind aber auch Überlagerungen von Zuständen
zugelassen. Kann eine klassische Münze beispielsweise nur einen
Zustand $p \in \left\lbrace \textrm{Kopf}, \textrm{Zahl}\right\rbrace$
einnehmen, so wird der Zustandsraum einer Quantenmünze durch den
Einheitskreis in dem freien Vektorraum über der Menge der klassischen
Zustände beschrieben, in diesem Fall beispielsweise durch $\psi \in
\textrm{span} \left\lbrace \textrm{Kopf}, \textrm{Zahl}\right\rbrace$,
oder in QM Notation $\psi = a_1 \ket{\textrm{Kopf}} + a_2 \ket{\textrm{Zahl}}$,
wobei die Nebenbedingung $\sum_i a^ 2_i = 1$ meist impliziert wird.

In der Quantenfeldtheorie (QFT) werden QM Zustände dieser Art jedem
Raumpunkt zugeordnet. Diese Zuordnungen werden in einem physikalischen
Kontext Felder genannt, wobei beispielsweise für jeden bekannte Sorte
fundamentaler Teilchen ein solches Feld deklariert und die Präsenz
eines entsprechenden Teilchens an einem Raumpunkt durch eine
sogenannte \glqq Anregung\grqq{} des entsprechenden Feldes modelliert
wird. Eine solche Anregung lässt sich am einfachsten verstehen, indem
man sich das Feld an jedem Punkt als schwingungsfähige Saite eines
Instruments visualisiert, wobei der Grad der Anregung einer bestimmten
Schwingungsmode zugeordnet wird. Prinzipiell kann ein Feld an einem
Raumpunkt aber beliebigen Wert $\psi\in\mathbb{C}$ (oder $\mathbb{R}$,
je nach Art des Feldes) annehmen.

Die für diesen Mechanismus implizierte Annahme ist, dass der
Raumbereiche über eine feste, zeitlich konstante und trivial messbare
Größe verfügen, was von der ART nicht erfüllt wird. Stattdessen
verwendet diese das Konzept einer verformbaren Raumzeit, welche aus
einer Zusammenführung der drei räumlichen mit einer zeitlichen
Dimension zu einer vierdimensionalen Mannigfaltigkeit mit
gegebenenfalls gekrümmter Geometrie entsteht.

\end{siderule}
%% Mannigfaltigkeiten, Kobordismen (Max)
%% Pfadintegral, Wirkung (Carsten)
\begin{siderule}
Um von einem bestimmten klassischen Zustand $a$ am Zeitpunkt $t_i$ zu
einem anderen klassischen Zustand $b$ zum Zeitpunkt $t_f > t_i$ zu
gelangen, muss ein System eine Menge anderer Zustände in zeitlicher
Abfolge durchlaufen, die eine sogenannte Trajektorie $\phi(t)$
bilden. Typischerweise existieren jedoch viele verschiedene
Trajektorien für einen bestimmten Übergang: Der Übergang eines
Fußballes von der Hand des Schiedsrichters ins Tornetz kann
beispielsweise auf viele verschiedene Weisen erfolgen. In der
klassischen Physik wird davon ausgegangen, dass von diesen Pfaden nur
ein einziger realisiert wird. Der von einem System realisierte Pfad
ist dabei stets einer, welcher die \textit{Wirkung} $S(\phi)$ extremal
werden lässt. 
\begin{align}
  S\left(\phi\right) &= \int_{\mathbb{R}^n} \Ell\left(\phi\left(x\right),\del \phi\left(x\right) \right) d^nx
\end{align}

In der QM lässt sich der gleiche Mechanismus verwenden, jedoch kann
hier nicht die Rede davon sein, dass nur ein bestimmter Pfad
realisiert wird -- stattdessen befindet sich das unbeobachtete System
in der Zwischenzeit $t_i < t < t_f$ in einer Überlagerung aller in
Frage kommenden Zustände. Mathematisch ist dies zum Zeitpunkt $t$ eine
Linearkombination der Zustände aller in Frage kommenden Pfade. Als
Koeffizienten werden hierbei die Wirkungen der einzelnen Pfade
verwendet. Falls die Menge der in Frage kommenden Pfade $\phi$
überabzählbar ist, lässt sich diese Linearkombination nurnoch mithilfe
eines Maßes $D\phi${} auf dem Raum, in dem sie leben,
definieren. Räume, auf denen dies möglich ist, heißen messbar.

Für Pfadintegrale in der Quantenfeldtheorie lässt sich ein solches Maß
allerdings nicht mehr rigoros definieren, und nur noch in einigen
Fällen, nur Näherungsweise und nur unter ignorieren diverser
mathematischer Probleme gelöst werden. Im Prinzip jedoch wäre
beispielsweise die Wahrscheinlichkeit eines Übergangs
$\phi_i\to\phi_f$ berechenbar über
\begin{align*}
 \bra{\phi_f=\phi(t_f)}
   \widehat{U}\left(t_i,t_f\right)
 \ket{\phi_i=\phi(t_i)}
  &= \int_{\phi} S\left(\phi\right) \glqq D\phi\grqq{}
\end{align*}

\end{siderule}
%% 1-Kategorien (Max)
%% TQFTs (Jochen)



Um von erweiterten topologischen Quantenfeldtheorien (ETQFT) reden zu können,
muss man natürlich zunächst wissen, was topologische Quantenfeldtheorien (TQFT) sind.

In der allgemeinen Relativitätstheorie wird die Raumzeit als Mannigfaltigkeit dargestellt.
Sei nun $Z(M)$ die Überlagerung der quantenmechanischen Zustände auf der
$d$-dimensionalen Mannigfaltigkeit $M$.

% Darstellung der TQFTs als Funktoren von der Kategorie der Mannigfaltigkeiten in die Kategorie der C-VR
% Isomorphismen sind wohl die Diffeomorphismen,was sind aber die nicht-Isomorphismen?
% TODO



%% Klassifkikation von 1-TQFTs/2-TQFTs (Leon)
%% Zerschneiden, Lokalität, Zustandssummen (Carsten)
\begin{siderule}
Die konstruierten TQFTs erfüllen viele wünschenswerte Eigenschaften,
sie sind jedoch noch nicht zwangsläufig \textit{lokal}. Lokalität ist ein
physikalisches Konzept, das anschaulich wie folgt interpretiert werden
kann: Die physikalischen Vorgänge an einem bestimmten Punkt sollten
auf kurzen Zeitskalen nur von der unmittelbaren Umgebung dieses
Punktes abhängen. Dies verbietet spukhafte Fernwirkungen wie Albert
Einstein sie bei Einführung der QM fürchtete und erzwingt außerdem,
dass die Axiome der Relativitätstheorie gewahrt bleiben und
insbesondere die Lichtgeschwindigkeit als maximale Geschwindigkeit für
Informationstransfer nicht überschritten werden kann.

Für die hier betrachteten Kobordismen kann Lokalität insbesondere
dadurch erzwungen werden, dass man es erlaubt, die Raumzeit nicht nur
entlang eines raumartigen Randes, sondern entlang beliebiger
Raumzeitränder zu zerschneiden.

Ferner ist das zu Beginn angesprochene Problem eines undefinierten
Maßes auf dem Raum der möglichen Pfade noch immer nicht
zufriedenstellend gelöst -- eine Lösung lässt sich aber basierend auf
der bereits geleisteten Vorarbeit leicht realisieren: Die Kobordismen
können zwischen zwei raumartigen Rändern diskretisiert werden, sodass
nur noch eine endliche Menge an Pfaden existiert. Am einfachsten lässt
sich eine solche Diskretisierung durch eine Pflasterung mit
Simplices erreichen. Für zweidimensionale Kobordismen sind dies
Dreiecke, weshalb man auch von einer Triangulierung spricht.

Diese Diskretisierung entspricht physikalisch einer Quantisierung der
Raumzeit. Naheliegenderweise würde man in einer physikalischen Theorie
als Seitenlänge beispielsweise die Planck-Länge
wählen. Erfreulicherweise erzwingt die algebraische Struktur der
Theorie aber, dass das Ergebnis einer Summation über alle Pfade
unabhängig von der gewählten Triangulierung ist.
%%%TODO: stimmt das? (carsten)


\end{siderule}
%% Höhere Kategorien (Tashi)
\begin{Def}
Eine \emph{Bikategorie} $\mathfrak C$ besteht aus:
\begin{itemize}
\item einer Klasse von Objekten $\mathrm{ob}(\mathfrak C)$,
\item f"ur je zwei Objekte $X$ und $Y$ von $\mathfrak C$ einer ($1$-)Kategorie $\mathfrak{C}(X,Y)$ von Pfeilen,
\item einer Komposition, d.h.\ einem Funktor \[-\circ-\colon \mathfrak{C}(X,Y)\times \mathfrak{C}(Y,Z)\longrightarrow\mathfrak{C}(X,Z)\]
f"ur je drei Objekte $X$, $Y$ und $Z$ von $\mathfrak C$ und
\item f"ur jedes Objekt $X$ von $\mathfrak C$ aus einem Identit"atspfeil $1_X\in \mathrm{ob}\mathfrak C(X,X)$.% Wir fassen $1_X$ als einen Funktor $\star\to\mathfrak C(X,X)$ auf, wobei $\star$ die diskrete Kategorie mit einem Objekt ist.
\end{itemize}
Wir fordern die "ublichen Axiome (Unitarit"at $-\circ 1\cong \mathrm{id}$, $1\circ -\cong \mathrm{id}$ und Assoziativit"at $(-\circ-)\circ-\cong -\circ(-\circ-)$), allerdings nur bis auf nat"urlichen Isomorphisms. Diese Isomorphismen sollen selbst wieder gewisse Koh"arenzbedingungen erf"ullen, die wir hier nicht ausf"uhrlich hinschreiben wollen.\\
Die Objekte bzw. Morphismen der Pfeilkategorien $\mathfrak C(-,-)$ nennen wir auch \emph{1-Morphismen} bzw. \emph{2-Morphismen} von $\mathfrak C$.
\end{Def}
Es gibt nat"urlich auch den Begriff eines Bifunktors zwischen Bikategorien. Es handelt sich dabei um eine Abbildung auf Objekten und einem Funktor auf jeder Pfeilkategorie, die kompatibel sind mit den Assoziativit"ats- und Unitarit"atsisomorphismen. Wir "uberlassen es dem Leser eine detaillierte Definition auszuarbeiten.

\begin{Def}
Ein \emph{2-Vektorraum} ("uber einem K"orper $k$) $\mathfrak V$ ist eine "uber $\category{Vekt}_k$ angereicherte Kategorie (d.h.\ die Pfeilmengen $\mathfrak V(x,y)$ sind $k$-Vektorr"aume und die Komposition ist bilinear), die folgende Eigenschaften erf"ullt:
\begin{itemize}
\item $\mathfrak V$ enth"alt ein $0$-Objekt,
\item in $\mathfrak V$ gibt es bin"are Produkte und Coprodukte und diese stimmen "uberein (wir nennen diese Produkte/Coprodukte auch direkte Summen) und
\item $\mathfrak V$ l"asst eine Basis der Objekte zu.
\end{itemize}
Eine Klasse $B$ von Objekten in $\mathfrak V$ hei\ss t Basis, falls:
\begin{itemize}
\item $\mathfrak V(b,c) \cong\begin{cases}
  k,  & \text{wenn }c=b\\
  0, & \text{wenn }c\neq b
\end{cases}$ f"ur $b,c \in B$ und
\item jedes Objekt in $\mathfrak V$ eine bis auf Permutation eindeutige Darstellung der Form $\bigoplus\limits_{b\in B} b^{\oplus n_b}$ besitzt, wobei $n_b$ nur f"ur endlich viele $b\in B$ nicht null ist.
\end{itemize}
\end{Def}

\begin{Bsp}
Die Kategorie $\category{Vekt}_k$ ist ein 2-Vektorraum mit Basis $\{k\}$.
\end{Bsp}

Nun definieren wir die Bikategorie $\category{2Vect}_k$ der 2-Vektorr"aume. Was sind also die Morphismen zwischen zwei 2-Vektorr"aumen $\mathfrak V$ und $\mathfrak W$? Da es sich um Kategorien mit Zusatzeigenschaften handelt, ist es naheliegend einfach die Funktoren zu nehmen, die diese Zusatzeigenschaften respektieren.\\
Die Objekte der Morphismenkategorie $\category{2Vect}_k(\mathfrak V,\mathfrak W)$ sind also  diejenigen Funktoren $F\colon \mathfrak V\to \mathfrak W$, die auf den Pfeilmengen lineare Abbildungen $\mathfrak V(-,-)\to\mathfrak W(-,-)$ induzieren und mit (endlichen) direkten Summen kommutieren (d.h.\ $F\left(\bigoplus_i x_i\right)=\bigoplus_i F(x_i)$). Wir nennen solche Funktoren \emph{linear}.\\
Die 2-Morphismen von $\category{2Vect}_k$ sind nichts anderes als nat"urliche Transformationen von Funktoren.

\begin{Bsp}
Die Endomorphismenkategorie $\category{2Vect}_k(\category{Vekt}_k,\category{Vekt}_k)$ ist durch den kanonischen Funktor ($F\mapsto F(k)$ auf Objekten) equivalent zur Kategorie $\category{Vekt}_k$ der $k$-Vektorr"aume.
\end{Bsp}

Die letzte Zutat, die wir brauchen um ETQFTs zu definieren, ist ein ``Tensorprodukt'' von 2-Vektorr"aumen. F"ur zwei 2-Vektorr"aume $\mathfrak V$ und $\mathfrak W$, ist $\mathfrak V\otimes\mathfrak W$ der eindeutige (bis auf Isomorphie) 2-Vektorraum sodass bilineare Funktoren $\mathfrak V\times\mathfrak W\longrightarrow ?$ genau linearen Funktoren $\mathfrak V\otimes\mathfrak W\longrightarrow ?$ entsprechen. Nachdem man sich "uberzeugt hat, dass 2-Vektorr"aume (wie ihre niedrigdimensionalen Geschwister) bis auf Isomorphie durch eine Basis festgelegt sind, kann man $\mathfrak V\otimes\mathfrak W$ als den 2-Vektorraum mit Basis $B\times C$ konstruieren, wenn man Basen $B$ und $C$ von $\mathfrak V$ bzw. $\mathfrak W$ gew"ahlt hat.\\

%\begin{Bsp}
%F"ur jeden 2-Vektorraum $\mathfrak V$ gilt $\mathfrak V\otimes \category{Vekt}_k\cong\mathfrak{V}$; der 2-Vektorraum $\category{Vekt}_k$ ist also (analog zum Vektorraum $k$) ein neutrales Element f"ur $\otimes$.
%\end{Bsp}

Das oben definierte Tensorprodukt liefert einen Bifunktor
\[\otimes\colon \category{2Vekt}_k\otimes\category{2Vekt}_k\longrightarrow \category{2Vekt}_k\]
Dieses Tensorprodukt ist (bis auf gewisse nat"urliche Isomorphismen) assoziativ, kommutativ und hat Einheit $\category{Vekt}_k$; es macht demnach $\category{2Vekt}_k$ zu einer \emph{symmetrisch monoidalen Bikategorie}.
%
%
%
%
%
%
%
%% ETQFTs (Aras)


\end{document}
