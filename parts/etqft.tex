
% Wenn hier ein Titel stehen soll, kann er
% \subsection{erweiterte topologische Quantenfeldtheorien}
% heissen.

Nun wollen wir die höhere Kobordismenkategorie $\category{Cob}_{2, 0}$
definieren, die als Verallgemeinerung von $\category{Cob}_{1, 0}$ und
$\category{Cob}_{2, 1}$ der Definitionsbereich von ETQFTs sein soll. Dafür
brauchen wir allerdings noch einen Hilfsbegriff:

\begin{Def}
Eine $\left<2\right>$-Mannigfaltigkeit $\Sigma$, eine (orientierte)
''Mannigfaltigkeit'', die offene Teilmengen von $\mathbb{R}_{\geq 0}^2$ als
Kartengebiete hat.

Für ein solches $\Sigma$ sei $\del_1 \Sigma$ (bzw. $\del_2 \Sigma$) die
Menge aller Punkte, die unter den Kartenabbildungen auf einen Punkt von
$\mathbb{R}_{\geq 0} \times \{0\}$ (bzw. von $\{0\} \times
\mathbb{R}_{\geq 0}$) abgebildet werden.
\end{Def}

Wie die Zahl $2$ in dem Namen verrät, kann man auch
\emph{$\left<n\right>$-Mannigfaltigkeiten} definieren, die in höherdimensonalen
ETQFTs eine Rolle spielen. Wir werden uns aber zunächst auf den
zweidimensionalen Fall konzentrieren.

\begin{Def}
$\category{Cob}_{2, 0}$ ist die symmetrisch monoidale Bikategorie, die durch
die folgenden Daten (und die dazu passenden natürlichen Abbildungen) gegeben
ist:
\begin{itemize}
\item Die Objekte sind (kompakte, orientierte) $0$-Mannigfaltigkeiten.
\item Die $1$-Morphismen sind Kobordismen zwischen $0$-Mannigfaltigkeiten
(diesmal \emph{nicht} bis auf Diffeomorphie). Die Komposition ist wieder durch
Verklebung gegeben.
\item Die $2$-Morphismen in $\category{Cob}_{2, 0} (X_1, X_2) (M_1, M_2)$ sind
$\left<2\right>$-Mannigfaltigkeiten mit Isomorphismen $\del_1 \Sigma \cong M_1
\sqcup \overline{M_2}$ und $\del_2 \Sigma \cong I \times X_1 \sqcup I \times
\overline{X_2}$. Diese kann man ''entlang einer Hälfte von $\del_1$''
(''vertikal'') oder ''entlang einer Hälfte von $\del_2$'' (''horizontal'')
verkleben, was die höheren Kompositionen liefert.
\item Die monoidale Struktur ist wieder durch die disjunkte Vereinigung
$\sqcup$ von Mannigfaltigkeiten gegeben.
\end{itemize}
\end{Def}

\begin{Def}
Eine $(2, 0)$-erweiterte topologische Quantenfeldtheorie ($(2, 0)$-ETQFT) ist
ein symmetrisch monoidaler Bifunktor
\begin{equation}
Z: \category{Cob}_{2, 0} \to \category{2Vect}_k, \nonumber
\end{equation}
d. h. ein Bifunktor, der ''$\sqcup$ auf $\otimes$ abbildet'' (und
gewissermaßen mit den entsprechenden natürlichen Transformationen dieser
Operationen verträglich ist).
\end{Def}

Da $\emptyset$ bzw. $\category{Vect}_k$ ''das neutrale Objekt'' bezüglich
$\sqcup$ bzw. $\otimes$ ist, muss für ein ETQFT $Z(\emptyset) =
\category{Vect}_k$ gelten. Somit erhält man einen Funktor
\begin{equation}
\category{Cob}_{2, 0} (\emptyset, \emptyset) \to \category{2Vect}_k
(\category{Vect}_k, \category{Vect}_k). \nonumber
\end{equation}
Es ist aber $\category{Cob}_{2, 0} (\emptyset, \emptyset) \simeq
\category{Cob}_{2, 1}$ (die Objekte sind $1$-Mannigfaltigkeiten ohne Rand,
die Morphismen entsprechen Kobordismen) bzw. $\category{2Vect}_k
(\category{Vect}_k, \category{Vect}_k) \simeq \category{Vect}_k$ (s. o.).
Auf diese Weise liefert uns jede ETQFT eine zweidimensionale TQFT.

% Der Rest kann weggelassen oder radikal abgekuerzt werden.

Man kann nun in einer (höheren) monoidalen Kategorie $(\mathcal{C},
\boxtimes, I, \ldots)$ von \emph{dualisierbaren Objekten} reden, was im
eindimensionalen Fall solche Objekte $X$ sind, die zusammen mit einem weiteren
Objekt $Y$ und Morpismen $e: X \boxtimes Y \to I$ bzw. $h: I \to Y \boxtimes
X$ gegeben sind, die gewisse Axiome erfüllen. Zum Beispiel wird eine
eindimensionale TQFT $Z: \category{Cob}_{1, 0} \to \category{Vect}_k$ durch die
Wahl von einem dualisierbaren Objekt $Z(\mathrm{pt}^+)$ in $(\category{Vect}_k,
\otimes, \ldots)$ eindeutig festgelegt.

In höheren Kategorien kann man noch fordern, dass die in der Definition von
Dualisierbarkeit vorkommenden Morphismen und deren
''Dualisierbarkeitsmorphismen'' usw. (als Objekte von Morphismenkategorien)
ähnliche Dualisierbarkeitsbedingungen erfüllen, was den Begriff der
\emph{vollständigen Dualisierbarkeit} liefert. Dadurch erhält man die
folgende Beschreibung von ETQFTs, für die eine Beweisskizze bereits
existiert:

\begin{Sat}[Kobordismushypothese]
Eine $(n, 0)$-erweiterte TQFT $Z$ ist durch ein vollständig dualisierbares
Objekt $Z(\mathrm{pt}^+)$ in ihrem Wertebereich eindeutig festgelegt.
\end{Sat}
