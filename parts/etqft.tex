
Nun wollen wir die h\"ohere Kobordismenkategorie $\category{Cob}_{2, 0}$
definieren, die als Verallgemeinerung von $\category{Cob}_{1, 0}$ bzw.\
$\category{Cob}_{2, 1}$ der Definitionsbereich von ETQFTs sein soll. Daf\"ur
brauchen wir allerdings noch einen Hilfsbegriff:

\begin{Def}
Eine $\left<2\right>$-Mannigfaltigkeit $\Sigma$, eine (orientierte)
"`Mannigfaltigkeit"', die offene Teilmengen von $\mathbb{R}_{\geq 0}^2$ als
Kartengebiete hat.

F\"ur ein solches $\Sigma$ sei $\del_1 \Sigma$ (bzw.\ $\del_2 \Sigma$) die
Menge aller Punkte, die unter den Kartenabbildungen auf einen Punkt von
$\mathbb{R}_{\geq 0} \times \{0\}$ (bzw.\ von $\{0\} \times
\mathbb{R}_{\geq 0}$) abgebildet werden.
\end{Def}

Wie die Zahl $2$ in dem Namen verr\"at, kann man auch
$\left<n\right>$-Mannigfaltigkeit definieren, die in h\"oherdimensonalen ETQFTs
eine Rolle spielen. Wir werden uns aber zun\"achst auf den zweidimensionalen
Fall konzentrieren.

\begin{Def}
$\category{Cob}_{2, 0}$ ist die symmetrisch monoidale Bikategorie, die durch
die folgenden Daten (und die dazu passenden nat\"urlichen Abbildungen) gegeben
ist:
\begin{itemize}
\item Die Objekte sind (kompakte, orientierte) $0$-Mannigfaltigkeiten.
\item Die $1$-Morphismen sind Kobordismen zwischen $0$-Mannigfaltigkeiten
(diesmal \emph{nicht} bis auf Diffeomorphie). Die Komposition ist wieder durch
Verklebung gegeben.
\item Die $2$-Morphismen in $\category{Cob}_{2, 0} (X_1, X_2) (M_1, M_2)$ sind
$\left<2\right>$-Mannigfaltigkeiten mit Isomorphismen $\del_1 \Sigma \cong M_1
\sqcup \overline{M_2}$ und $\del_2 \Sigma \cong I \times X_1 \sqcup I \times
X_2$. Diese kann man entlang $\del_1$ ("`vertikal"') oder $\del_2$
("`horizontal"') verkleben, was die h\"oheren Kompositionen liefert.
\item Die monoidale Struktur ist wieder durch die disjunkte Vereinigung
$\sqcup$ von Mannigfaltigkeiten gegeben.
\end{itemize}
\end{Def}

\begin{Def}
Eine $(2, 0)$-erweiterte topologische Quantenfeldtheorie ($(2, 0)$-ETQFT) ist
ein symmetrisch monoidaler Bifunktor
\begin{equation}
Z: \category{Cob}_{2, 0} \to \category{2Vect}_k, \nonumber
\end{equation}
d.\,h.\ ein Bifunktor, der "`$\sqcup$ auf $\otimes$ abbildet"' (und
gewisserma\ss en mit den entsprechenden nat\"urlichen Transformationen dieser
Operationen vertr\"aglich ist).
\end{Def}

Da $\emptyset$ bzw.\ $\category{Vect}_k$ "`das neutrale Objekt"' bez\"uglich
$\sqcup$ bzw.\ $\otimes$ ist, muss f\"ur ein ETQFT $Z(\emptyset) =
\category{Vect}_k$ gelten. Somit erh\"alt man einen Funktor
\begin{equation}
\category{Cob}_{2, 0} (\emptyset, \emptyset) \to \category{2Vect}_k
(\category{Vect}_k, \category{Vect}_k). \nonumber
\end{equation}
Es ist aber $\category{Cob}_{2, 0} (\emptyset, \emptyset) \simeq
\category{Cob}_{2, 1}$ (die Objekte sind $1$-Mannigfaltigkeiten ohne Rand,
die Morphismen entsprechen Kobordismen) bzw.\ $\category{2Vect}_k
(\category{Vect}_k, \category{Vect}_k) \simeq \category{Vect}_k$  (s.\,o.).
Also liefert uns jede ETQFT auf diese Weise eine zweidimensionale TQFT.

Man kann nun in einer (h\"oheren) monoidalen Kategorie $(\category{C},
\boxtimes, I, \ldots)$ von \emph{dualisierbaren Objekten} reden, was im
eindimensionalen Fall solche Objekte $X$ sind, die zusammen mit einem weiteren
Objekt $Y$ und Morpismen $e: X \boxtimes Y \to I$ bzw.\ $h: I \to Y \boxtimes
X$ gegeben sind, die gewisse Axiome erf\"ullen. Zum Beispiel wird eine
eindimensionale TQFT $Z: \category{Cob}_{1, 0} \to \category{Vect}_k$ durch die
Wahl von einem dualisierbaren Objekt $Z(\mathrm{pt}^+)$ in $(\category{Vect}_k,
\otimes, \ldots)$ eindeutig festgelegt.

In h\"oheren Kategorien kann man noch fordern, dass die in der Definition von
Dualisierbarkeit vorkommenden Morphismen (als Objekte von Morphismenkategorien)
und deren "`Dualisierbarkeitsmorphismen"' usw.\  dualisierbar sind, was den
Begriff der \emph{vollst\"andigen Dualisierbarkeit} liefert. Somit erh\"alt man
die folgende Beschreibung von ETQFTs, f\"ur die eine Beweisskizze bereits
existiert:

\begin{Sat}[Kobordismushypothese]
Eine $(n, 0)$-erweiterte TQFT $Z$ ist durch ein vollst\"andig dualisierbares
Objekt $Z(\mathrm{pt}^+)$ in ihrem Wertebereich eindeutig festgelegt.
\end{Sat}
