\begin{Def}
Eine \emph{Bikategorie} $\mathfrak C$ besteht aus:
\begin{itemize}
\item einer Klasse von Objekten $\mathrm{ob}(\mathfrak C)$,
\item f"ur je zwei Objekte $X$ und $Y$ von $\mathfrak C$ einer ($1$-)Kategorie $\mathfrak{C}(X,Y)$ von Pfeilen,
\item einer Komposition, d.h.\ einem Funktor \[-\circ-\colon \mathfrak{C}(X,Y)\times \mathfrak{C}(Y,Z)\longrightarrow\mathfrak{C}(X,Z)\]
f"ur je drei Objekte $X$, $Y$ und $Z$ von $\mathfrak C$ und
\item f"ur jedes Objekt $X$ von $\mathfrak C$ aus einem Identit"atspfeil $1_X\in \mathrm{ob}\mathfrak C(X,X)$.% Wir fassen $1_X$ als einen Funktor $\star\to\mathfrak C(X,X)$ auf, wobei $\star$ die diskrete Kategorie mit einem Objekt ist.
\end{itemize}
Wir fordern die "ublichen Axiome (Unitarit"at $-\circ 1\cong \mathrm{id}$, $1\circ -\cong \mathrm{id}$ und Assoziativit"at $(-\circ-)\circ-\cong -\circ(-\circ-)$), allerdings nur bis auf nat"urlichen Isomorphisms. Diese Isomorphismen sollen selbst wieder gewisse Koh"arenzbedingungen erf"ullen, die wir hier nicht ausf"uhrlich hinschreiben wollen.\\
Die Objekte bzw. Morphismen der Pfeilkategorien $\mathfrak C(-,-)$ nennen wir auch \emph{1-Morphismen} bzw. \emph{2-Morphismen} von $\mathfrak C$.
\end{Def}
Es gibt nat"urlich auch den Begriff eines Bifunktors zwischen Bikategorien. Es handelt sich dabei um eine Abbildung auf Objekten und einem Funktor auf jeder Pfeilkategorie, die kompatibel sind mit den Assoziativit"ats- und Unitarit"atsisomorphismen. Wir "uberlassen es dem Leser eine detaillierte Definition auszuarbeiten.

\begin{Def}
Ein \emph{2-Vektorraum} ("uber einem K"orper $k$) $\mathfrak V$ ist eine "uber $\category{Vekt}_k$ angereicherte Kategorie (d.h.\ die Pfeilmengen $\mathfrak V(x,y)$ sind $k$-Vektorr"aume und die Komposition ist bilinear), die folgende Eigenschaften erf"ullt:
\begin{itemize}
\item $\mathfrak V$ enth"alt ein $0$-Objekt,
\item in $\mathfrak V$ gibt es bin"are Produkte und Coprodukte und diese stimmen "uberein (wir nennen diese Produkte/Coprodukte auch direkte Summen) und
\item $\mathfrak V$ l"asst eine Basis der Objekte zu.
\end{itemize}
Eine Klasse $B$ von Objekten in $\mathfrak V$ hei\ss t Basis, falls:
\begin{itemize}
\item $\mathfrak V(b,c) \cong\begin{cases}
  k,  & \text{wenn }c=b\\
  0, & \text{wenn }c\neq b
\end{cases}$ f"ur $b,c \in B$ und
\item jedes Objekt in $\mathfrak V$ eine bis auf Permutation eindeutige Darstellung der Form $\bigoplus\limits_{b\in B} b^{\oplus n_b}$ besitzt, wobei $n_b$ nur f"ur endlich viele $b\in B$ nicht null ist.
\end{itemize}
\end{Def}

\begin{Bsp}
Die Kategorie $\category{Vekt}_k$ ist ein 2-Vektorraum mit Basis $\{k\}$.
\end{Bsp}

Nun definieren wir die Bikategorie $\category{2Vect}_k$ der 2-Vektorr"aume. Was sind also die Morphismen zwischen zwei 2-Vektorr"aumen $\mathfrak V$ und $\mathfrak W$? Da es sich um Kategorien mit Zusatzeigenschaften handelt, ist es naheliegend einfach die Funktoren zu nehmen, die diese Zusatzeigenschaften respektieren.\\
Die Objekte der Morphismenkategorie $\category{2Vect}_k(\mathfrak V,\mathfrak W)$ sind also  diejenigen Funktoren $F\colon \mathfrak V\to \mathfrak W$, die auf den Pfeilmengen lineare Abbildungen $\mathfrak V(-,-)\to\mathfrak W(-,-)$ induzieren und mit (endlichen) direkten Summen kommutieren (d.h.\ $F\left(\bigoplus_i x_i\right)=\bigoplus_i F(x_i)$). Wir nennen solche Funktoren \emph{linear}.\\
Die 2-Morphismen von $\category{2Vect}_k$ sind nichts anderes als nat"urliche Transformationen von Funktoren.

\begin{Bsp}
Die Endomorphismenkategorie $\category{2Vect}_k(\category{Vekt}_k,\category{Vekt}_k)$ ist durch den kanonischen Funktor ($F\mapsto F(k)$ auf Objekten) equivalent zur Kategorie $\category{Vekt}_k$ der $k$-Vektorr"aume.
\end{Bsp}

Die letzte Zutat, die wir brauchen um ETQFTs zu definieren, ist ein ``Tensorprodukt'' von 2-Vektorr"aumen. F"ur zwei 2-Vektorr"aume $\mathfrak V$ und $\mathfrak W$, ist $\mathfrak V\otimes\mathfrak W$ der eindeutige (bis auf Isomorphie) 2-Vektorraum sodass bilineare Funktoren $\mathfrak V\times\mathfrak W\longrightarrow ?$ genau linearen Funktoren $\mathfrak V\otimes\mathfrak W\longrightarrow ?$ entsprechen. Nachdem man sich "uberzeugt hat, dass 2-Vektorr"aume (wie ihre niedrigdimensionalen Geschwister) bis auf Isomorphie durch eine Basis festgelegt sind, kann man $\mathfrak V\otimes\mathfrak W$ als den 2-Vektorraum mit Basis $B\times C$ konstruieren, wenn man Basen $B$ und $C$ von $\mathfrak V$ bzw. $\mathfrak W$ gew"ahlt hat.\\

%\begin{Bsp}
%F"ur jeden 2-Vektorraum $\mathfrak V$ gilt $\mathfrak V\otimes \category{Vekt}_k\cong\mathfrak{V}$; der 2-Vektorraum $\category{Vekt}_k$ ist also (analog zum Vektorraum $k$) ein neutrales Element f"ur $\otimes$.
%\end{Bsp}

Das oben definierte Tensorprodukt liefert einen Bifunktor
\[\otimes\colon \category{2Vekt}_k\otimes\category{2Vekt}_k\longrightarrow \category{2Vekt}_k\]
Dieses Tensorprodukt ist (bis auf gewisse nat"urliche Isomorphismen) assoziativ, kommutativ und hat Einheit $\category{Vekt}_k$; es macht demnach $\category{2Vekt}_k$ zu einer \emph{symmetrisch monoidalen Bikategorie}.
%
%
%
%
%
%
%