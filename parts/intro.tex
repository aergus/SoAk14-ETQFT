Den meisten wissenschaftlich interessierten Menschen sind sowohl die
Quantenmechanik als auch die (allgemeine oder spezielle)
Relativitätstheorie ein Begriff. Diese beiden physikalischen Theorien
verkörpern in gewissem Sinne physikalische Grenzwerte, welche der
Alltagserfahrung von der uns umgebenden Physik oft eklatant
widersprechen. Der Grund hierfür ist, dass sich die Quantenmechanik
mit dem Grenzfall sehr kleiner Distanzen und Zeitspannen beschäftigt,
auf welchen die Gravitation oft keine Rolle spielt. Die (allgemeine)
Relativitätstheorie auf der anderen Seite modelliert makroskopische,
gravitative Vorgänge. Die Überlagerung dieser beiden Grenzfälle, also
gravitative Effekte auf mikrokopischer Ebene, ist nicht nur
experimentell sehr schwer zu beobachten, sondern erfordert auf der
theoretischen Seite auch eine Auseinandersetzung mit der Frage, wie
zwei konzeptionell so unterschiedliche Theorien wie Quantenmechanik
(QM) und Allgemeine Relativität (ART) auf eine konsistente Art und
Weise zusammengeführt werden können. Dieses Problem hat sich als
ausgesprochen schwierig herausgestellt, und so gibt es zum heutigen
Tage mehrere Ansätze, eine erfolgreiche theoretische Beschreibung von
Quantengravitation zu etablieren. Ein verhältnismäßig neues Konzept
für die Lösung dieses alten Problems sind Topologische
Quantenfeldtheorien (TQFTs), welche im Folgenden erläutert werden
sollen.

Um in diesem mathematischen Artikel einer diversen Leserschaft gerecht
zu werden, werden einige Passagen mit seitlichen Linien markiert --
diese dienen zur Veranschaulichung und können von lediglich
mathematisch interessierten Lesern übersprungen werden, während
physikalisch interessierte sie hoffentlich hilfreichn finden werden
und lediglich an einer Anschauung interessierte Leser auch
ausschließlich die markierten Abschnitte lesen und die mathematischen
Passagen überspringen können,ohne dabei den Lesefluss zu behindern.
