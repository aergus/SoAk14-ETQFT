Um von einem bestimmten klassischen Zustand $a$ am Zeitpunkt $t_i$ zu
einem anderen klassischen Zustand $b$ zum Zeitpunkt $t_f > t_i$ zu
gelangen, muss ein System eine Menge anderer Zustände in zeitlicher
Abfolge durchlaufen, die eine sogenannte Trajektorie $\phi(t)$
bilden. Typischerweise existieren jedoch viele verschiedene
Trajektorien für einen bestimmten Übergang: Der Übergang eines
Fußballes von der Hand des Schiedsrichters ins Tornetz kann
beispielsweise auf viele verschiedene Weisen erfolgen. In der
klassischen Physik wird davon ausgegangen, dass von diesen Pfaden nur
ein einziger realisiert wird. Der von einem System realisierte Pfad
ist dabei stets einer, welcher die \textit{Wirkung} $S(\phi)$ extremal
werden lässt. 
\begin{align}
  S\left(\phi\right) &= \int_{\mathbb{R}^n} \Ell\left(\phi\left(x\right),\del \phi\left(x\right) \right) d^nx
\end{align}

In der QM lässt sich der gleiche Mechanismus verwenden, jedoch kann
hier nicht die Rede davon sein, dass nur ein bestimmter Pfad
realisiert wird -- stattdessen befindet sich das unbeobachtete System
in der Zwischenzeit $t_i < t < t_f$ in einer Überlagerung aller in
Frage kommenden Zustände. Mathematisch ist dies zum Zeitpunkt $t$ eine
Linearkombination der Zustände aller in Frage kommenden Pfade. Als
Koeffizienten werden hierbei die Wirkungen der einzelnen Pfade
verwendet. Falls die Menge der in Frage kommenden Pfade $\phi$
überabzählbar ist, lässt sich diese Linearkombination nurnoch mithilfe
eines Maßes $D\phi${} auf dem Raum, in dem sie leben,
definieren. Räume, auf denen dies möglich ist, heißen messbar.

Für Pfadintegrale in der Quantenfeldtheorie lässt sich ein solches Maß
allerdings nicht mehr rigoros definieren, und nur noch in einigen
Fällen, nur Näherungsweise und nur unter ignorieren diverser
mathematischer Probleme gelöst werden. Im Prinzip jedoch wäre
beispielsweise die Wahrscheinlichkeit eines Übergangs
$\phi_i\to\phi_f$ berechenbar über
\begin{align*}
 \bra{\phi_f=\phi(t_f)}
   \widehat{U}\left(t_i,t_f\right)
 \ket{\phi_i=\phi(t_i)}
  &= \int_{\phi} S\left(\phi\right) \glqq D\phi\grqq{}
\end{align*}
