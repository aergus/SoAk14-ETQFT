In der QM werden physikalische Zustände oft durch Vektoren eines
Hilbertraums beschrieben. Im Unterschied zu einer klassischen
Betrachtung sind aber auch Überlagerungen von Zuständen
zugelassen. Kann eine klassische Münze beispielsweise nur einen
Zustand $p \in \left\lbrace \textrm{Kopf}, \textrm{Zahl}\right\rbrace$
einnehmen, so wird der Zustandsraum einer Quantenmünze durch den
Einheitskreis in dem freien Vektorraum über der Menge der klassischen
Zustände beschrieben, in diesem Fall beispielsweise durch $\psi \in
\textrm{span} \left\lbrace \textrm{Kopf}, \textrm{Zahl}\right\rbrace$,
oder in QM Notation $\psi = a_1 \ket{\textrm{Kopf}} + a_2 \ket{\textrm{Zahl}}$,
wobei die Nebenbedingung $\sum_i a^ 2_i = 1$ meist impliziert wird.

In der Quantenfeldtheorie (QFT) werden QM Zustände dieser Art jedem
Raumpunkt zugeordnet. Diese Zuordnungen werden in einem physikalischen
Kontext Felder genannt, wobei beispielsweise für jeden bekannte Sorte
fundamentaler Teilchen ein solches Feld deklariert und die Präsenz
eines entsprechenden Teilchens an einem Raumpunkt durch eine
sogenannte \glqq Anregung\grqq{} des entsprechenden Feldes modelliert
wird. Eine solche Anregung lässt sich am einfachsten verstehen, indem
man sich das Feld an jedem Punkt als schwingungsfähige Saite eines
Instruments visualisiert, wobei der Grad der Anregung einer bestimmten
Schwingungsmode zugeordnet wird. Prinzipiell kann ein Feld an einem
Raumpunkt aber beliebigen Wert $\psi\in\mathbb{C}$ (oder $\mathbb{R}$,
je nach Art des Feldes) annehmen.

Die für diesen Mechanismus implizierte Annahme ist, dass der
Raumbereiche über eine feste, zeitlich konstante und trivial messbare
Größe verfügen, was von der ART nicht erfüllt wird. Stattdessen
verwendet diese das Konzept einer verformbaren Raumzeit, welche aus
einer Zusammenführung der drei räumlichen mit einer zeitlichen
Dimension zu einer vierdimensionalen Mannigfaltigkeit mit
gegebenenfalls gekrümmter Geometrie entsteht.
