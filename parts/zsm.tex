Die konstruierten TQFTs erfüllen viele wünschenswerte Eigenschaften,
sie sind jedoch noch nicht zwangsläufig \textit{lokal}. Lokalität ist ein
physikalisches Konzept, das anschaulich wie folgt interpretiert werden
kann: Die physikalischen Vorgänge an einem bestimmten Punkt sollten
auf kurzen Zeitskalen nur von der unmittelbaren Umgebung dieses
Punktes abhängen. Dies verbietet spukhafte Fernwirkungen wie Albert
Einstein sie bei Einführung der QM fürchtete und erzwingt außerdem,
dass die Axiome der Relativitätstheorie gewahrt bleiben und
insbesondere die Lichtgeschwindigkeit als maximale Geschwindigkeit für
Informationstransfer nicht überschritten werden kann.

Für die hier betrachteten Kobordismen kann Lokalität insbesondere
dadurch erzwungen werden, dass man es erlaubt, die Raumzeit nicht nur
entlang eines raumartigen Randes, sondern entlang beliebiger
Raumzeitränder zu zerschneiden.

Ferner ist das zu Beginn angesprochene Problem eines undefinierten
Maßes auf dem Raum der möglichen Pfade noch immer nicht
zufriedenstellend gelöst -- eine Lösung lässt sich aber basierend auf
der bereits geleisteten Vorarbeit leicht realisieren: Die Kobordismen
können zwischen zwei raumartigen Rändern diskretisiert werden, sodass
nur noch eine endliche Menge an Pfaden existiert. Am einfachsten lässt
sich eine solche Diskretisierung durch eine Pflasterung mit
Simplices erreichen. Für zweidimensionale Kobordismen sind dies
Dreiecke, weshalb man auch von einer Triangulierung spricht.

Diese Diskretisierung entspricht physikalisch einer Quantisierung der
Raumzeit. Naheliegenderweise würde man in einer physikalischen Theorie
als Seitenlänge beispielsweise die Planck-Länge
wählen. Erfreulicherweise erzwingt die algebraische Struktur der
Theorie aber, dass das Ergebnis einer Summation über alle Pfade
unabhängig von der gewählten Triangulierung ist.
%%%TODO: stimmt das? (carsten)

